%% exam_paper_template.tex
%% by Lawrence Ong
%% Created: 13/09/2019
%% Modified: 16/10/2019
%% 
%% This is a template for final examination papers
%% created according to the specifications required
%% by the University of Newcastle


\documentclass[a4paper,12pt]{article}

% ==========
% Insert semester, year, and module name here
\newcommand{\semester}{2} % semester
\newcommand{\examyear}{2020} % year
\newcommand{\subjectname}{Wireless Communication} % subject name
\newcommand{\quiznumber}{1} % quiz number
\newcommand{\modulecode}{ELEC4550/ELEC6550} % subject code
% ==========

\usepackage{helvet}
\renewcommand{\familydefault}{\sfdefault}

\usepackage[T1]{fontenc}
\usepackage{amssymb,amsmath}
\usepackage{bm}
\usepackage{lastpage}
\usepackage{color}

\usepackage{graphicx}
\graphicspath{{../figures/}}

\usepackage[shortlabels]{enumitem}
\usepackage{setspace}

\usepackage{fancyhdr}
\pagestyle{fancy}
\fancyhead[L]{\fontsize{10}{0} \selectfont Semester~\semester,~\examyear}
\fancyhead[R]{\fontsize{10}{0} \selectfont \hfill \modulecode~\subjectname}
\fancyfoot[C]{\fontsize{10}{12} \selectfont Page \thepage\ of \pageref{LastPage}}
\renewcommand{\headrulewidth}{0pt}
\renewcommand{\footrulewidth}{0pt}
\setlength{\voffset}{-1.2cm} % -1in + 1.25cm = -1.29cm
\setlength{\topmargin}{0in}
\setlength{\headheight}{8.85pt}
\setlength{\headsep}{1.12cm}
\setlength{\oddsidemargin}{0cm}
\setlength{\hoffset}{0.65cm} %-1in + 3.25cm = 0.71cm
\setlength{\textwidth}{15.25cm}
\setlength{\textheight}{24.5cm}
\setlength{\footskip}{1.12cm}
\fancyheadoffset[LO,RE]{1.5cm} %move start of header to the left
\fancyheadoffset[RO,LE]{1cm} %move end of header to the right

\setlength\parindent{0pt}

\clearpage
% Start counting from page 2. Cover page to be manually inserted.
%\setcounter{page}{2}

\newcounter{question}[section]
\newenvironment{question}[2][]
{\refstepcounter{question} %\par\medskip
  \textbf{Question~\thequestion. \hfill (#1 marks)}\newline #2}{\bigskip\bigskip\bigskip}

% Example of usage
% Note: The first argument in the "question" environment
% is the total mark for that question
% ===============
% %Make sure that there is no space before the % sign below
% \begin{question}[10]\label{q2}%
%   % Include \vspace{-1.5\topsep} if there is no text preceeding enumerate
%   \vspace*{-1.5\topsep}
%   \begin{enumerate}[(a)]
%   \item XXX
%     \begin{enumerate}[(i)]
%     \item Question~\ref{q2}
%     \end{enumerate}
%   \item YYY
%   \end{enumerate}
% \end{question}
% ===============
 

\begin{document}
Student Number: \rule{5cm}{0.15mm}

{\ }

Student Name: \rule{5.5cm}{0.15mm}

{\ }

{\ }

\begin{center}
  {\bfseries \large
    School of Electrical Engineering \& Computing

    {\ }

    QUIZ~\quiznumber

    {\ }
    
    \textnormal{Semester~\semester,~\examyear}

    {\ }
    
    \modulecode~\subjectname}
\end{center}

{\ }

\hrulefill

\begin{doublespace}
  Quiz Duration: 50 minutes
  
  This quiz has \underline{NINE~(9)} questions.
  
  Total marks: 100 marks
  
  \textbf{Quiz Conditions:}
  
  This is an OPEN-BOOK quiz.
  
  \textbf{Special Instructions:}
  
  Answer \underline{ALL} questions.
  
  You must show workings in your answers.
\end{doublespace}

\hrulefill

\newpage

%Make sure that there is no space before the % sign below
\begin{question}[10]\label{q1}% 
  XXX

  YYYy
  
  \begin{itemize}[\large $\square$]
  \item XXX
    \begin{enumerate}[(i)]
    \item Question~\ref{q1}
    \end{enumerate}
  \end{itemize}
\end{question}

\begin{question}[10]\label{q2}%
  % Include \vspace{-1.5\topsep} if there is no parapgraph preceeding enumerate
  \vspace*{-1.5\topsep}
  \begin{enumerate}[(a)]
  \item XXX
    \begin{enumerate}[(i)]
    \item Question~\ref{q2}
    \end{enumerate}
  \item YYY
  \end{enumerate}
\end{question}


      \newpage


\begin{question}[20]\label{q3}%
  XXX
  \begin{enumerate}[(a)]
  \item XXX
    \begin{enumerate}[(i)]
    \item Question~\ref{q3}
    \end{enumerate}
  \end{enumerate}
\end{question}

\begin{center}
  \textbf{END OF QUIZ}
\end{center}

\end{document}
